\chapter{Introduction}
\label{sec:introduction}
% \chapter{Einleitung}
% \label{sec:einleitung}

Frame-based cameras that widely used in computer vision output images
at a pre-set rate, even when the intensity values stay unchanged,
resulting redundant data. When the scene moves too fast, an
insufficient frame rate would cause motion blur. To address this
problem, \citet{lichtsteiner2008128} presents an event-based camera
called DVS (dynamic vision sensor) that reports pixel-level log
intensity changes at a rate of \unit{MHz} scale. Unlike traditional
cameras that outputs a frame until all the pixels are scanned, an
event-based camera outputs are asynchronous: when the intensity change
at a pixel reaches a threshold
$\mid\ln{I(t)}-\ln{I(t-\Delta t)}\mid>C$, the sensor outputs an
``event'' $e=\{x,y,t,p\}$, which includes the pixel coordinate
$(x, y)$, the timestamp $t$ and the polarity $p=\pm1$ indicating
positive or negative intensity change. The output is thus a stream of
events instead of frames. Later \citet{brandli2014240} presents DAVIS
(dynamic and active pixel vision sensor) that has additional APS
(active pixel sensor) circuits that provides absolute intensity
information, while offering DVS outputs with higher resolution
(240$\times180$ against $128\times128$), higher dynamic range (130 dB
against 120 dB) and lower latency (3 $\mu$s against 15 $\mu$s). Newer
sensors also provide color
information\citep{li2015design,moeys2018sensitive} or have higher
image resolution\citep{son20174}.

Since for an event-base camera there is no such thing as a frames,
feature detection and tracking algorithms that work well for standard
camera can not be directly applied to an event-based camera. There are
several works that try to adopt feature detection and tracking
pipelines. \citet{zhu2017event} aggregate a small number of
events to construct a frame, and perform Harris corner detector
\citep{harris1988combined} on the synthesized frames, then track the
features with implicit data association. The work of
\citet{tedaldi2016feature} detects features on the APS outputs, and
performs tracking with the DVS outputs. Instead of working with
frames, \citet{mueggler2017fast} detect corners directly in the
spatiotemporal event stream.

algorithnm called EVO which iss proposed by Rebecq et al, try to
define a set of keyframes and back project the events as rays through
space. The region of high intensity marks the region of 3D edge
candidate. The pose of the frames is provided via aligning the current
frme with the map using Lukas Kanade method.


This are the works of Gallego et al that are most relevant to the work
of my master thesis. 
 
 As shown in the above picture, Directly
aggregating a set of events within a temporal window woulld cause a
visual effect that is similar to motion blur, since the camera is
moving and each event is happened at a different camera pose. 
 
 In
their earlier work the method is only applicable in rotation only
scenes, where they compensate the rotation via finding the angular
velocity that generates the imege with the largest contrast. In their
later works they also make it possible for 6DoF mition estimation in
planar scene. 
 
 these are the kind of images that we are going to
see throughout this work. The dark pixels correspond to a negative
event, white pixels are positine evetns.


Real-time 3D reconstruction and 6 DoF tracking with events only
\citep{kim2016real,rebecq2017evo}
